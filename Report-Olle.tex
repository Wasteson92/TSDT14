\documentclass[10pt]{article}

\usepackage{amsmath}
\usepackage{mathtools}
\usepackage{graphicx}
\usepackage{epstopdf}
\usepackage[T1]{fontenc}
\usepackage[utf8]{inputenc}
\usepackage{tabu}
\usepackage{parskip}
\begin{document}
\begin{center}

{\huge{\textbf{Time-discrete Stochastic Signals}}}

\vspace{3em}
\Large
\textbf{Course: TSDT14}
\vspace{3em}

\begin{tabular}{l r r}
  
Olle Hynén Ulfsjöö & 940727-1890 & ollul666 \\
Emil Wasteson & 920512-3570 & emiwa048 
\end{tabular}

\end{center}

\newpage
\tableofcontents
\newpage

\section{Introduction}


\subsection{Study 1 – Modelling Signals}

\subsection{Study 2 – Improving Estimates}

\subsection{Study 3 – Non-LTI-systems}

\subsection{Study 4 – Special Operations}



\pagebreak

\section{Study 1 – Modelling Signals}
Throughout this chapter we will estimate the ACF and PSD of filtered white noise and compare the results with the theoretical output. The estimation of the ACF is done both by Bartlett's method and Blackman-Tukey's method. The PSD is estimated by a periodogram. Finally we analyze the results and comment on the suitability of the different estimation methods.

\subsection{Theoretical background}
Before we get to the matter in hand we need to introduce some theory. This chapter will cover the non-trivial theory necessary for following the reasoning in study 1. 

\subsubsection{Auto-Correlation Function, ACF}
The Auto-Correlation Function (ACF) of a time-discrete process $X[n]$ is denoted $r_X[n_1,n_2]$ and defined by
\begin{equation*}
  r_X[n_1,n_2] = \text{E}\left\{X[n_1]X[n_2]\right\}.
\end{equation*}

\subsubsection{Power Spectral Density, PSD}
The Power Spectral Density (PSD) of a process is the Fourier transform of its ACF. 

\subsubsection{Wide-Sense Stationarity (WSS)}
A time-discrete process $X[n]$ is called Wide-Sense Stationary if the following holds
\begin{alignat*}{2}
  r_X[n_1,n_2] &= r_X[0,n_2-n_1] \qquad &&\forall \, n_1\, \text{and}\, n_2, \\
  m_X[n] &= m_X[0] &&\forall \,n.
\end{alignat*}
Then we simply denote the ACF by $r_X[k]$ where $k=n_2-n_1$ and the mean by $m_X$.

\subsubsection{White Noise}
A process, $X[n]$, with a constant PDF is said to be a white process or called white noise. That is, $R_X[\theta] = R_0$ for some non-zero constant $R_0$. In turn, this yields the ACF $r_X[k] = \delta[k] R_0$ where $\delta[k]$ is a time-discrete impulse located at $k$. 



\subsection{Results}

In this study we will consider white noise filtered through two different low-pass filters. The first low-pass filter is a first order moving average. It is nowhere close to an ideal filter but results in simple calucations for the theoretical cases. The second filter is low-pass filter of high degree that is apprixmated as an ideal filter.

The estimation of ACFs and PSDs will be done using three different methods. First off we estimate the ACF with Blackman-Tukey's estimate
\begin{equation*}
  \hat{r}_X[k] = \begin{cases}
    \frac{1}{N-|k|} \sum\limits_{n=0}^{N-|k|-1} x[n+|k|]]x[n], & |k| < N, \\
    0, & \text{elsewhere},
  \end{cases}
\end{equation*}
where $\left\{x[0],\dots,x[N-1]\right\}$ are our samples. This estimate has the attractive property of being unbiased but its problem lies with the variance. For the special case of the last sample, $k=N-1$, we have the variance of the estimate 
\begin{equation*}
  \sigma_{\hat{r}_X}^2[N-1] = r_X^2[0] + r_X^2[N-1] \rightarrow r_X^2[0] \neq 0\quad \text{as}\quad N \rightarrow \infty
\end{equation*}
since $r_X[\cdot]$ is usually small for large arguments. This is a problem because it may violate the property $|r_X[k]|\leq r_X[0]$ that much hold for all $k$ for an ACF. 

Thereafter we estimate the ACF with Bartlett's estimate
\begin{equation*}
  \hat{r}_X[k] = \begin{cases}
    \frac{1}{N}\sum\limits_{n=0}^{N-|k|-1} x[n+|k|]x[n], & |k| < N, \\
    0, & \text{elsewhere},
  \end{cases}
\end{equation*}
where, yet again, $\left\{x[0],\dots,x[N-1]\right\}$ are our samples. This estimate is biased but asymptotically unbiased for a fixed $k$. Especially its variance tends to zero most often for all $k$ including $k = N-1$. This difference between the two estimates will be revisited later on.

Finally we estimate the PSD by a periodogram,
\begin{equation*}
  \hat{R}_X[\theta] = \frac{1}{N}|X_N[\theta]|^2
\end{equation*}
where $X_N[\theta]$ is the Fourier transform of the observed sequence $x[n]$. This estimate is asymptotically unbiased and has a large variance. 


\subsection{Conclusions}


\clearpage



\section{Study 2 – Improving Estimates}


\subsection{Theoretical background}


\subsection{Results}


\subsection{Conclusions}


\clearpage


\section{Study 3 – Non-LTI-systems}


\subsection{Theoretical background}


\subsection{Results}


\subsection{Conclusions}


\clearpage


\section{Study 4 – Special Operations}


\subsection{Theoretical background}


\subsection{Results}


\subsection{Conclusions}


\end{document}
